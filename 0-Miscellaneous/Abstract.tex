






\chapter*{Abstract}
\addcontentsline{toc}{chapter}{Abstract}

\newlength{\abstskip}
\setlength{\abstskip}{8pt}

This thesis investigates the dynamics of electrons ionized by strong low frequency laser fields, from a semiclassical perspective, developing a trajectory-based formalism to describe the interactions of the outgoing electron with the remaining ion.

\vspace{\abstskip}
\noindent
Trajectory models for photoionization generally arise in the regime known as optical tunnelling, where the atom is subjected to a strong, slow field, which tilts the potential landscape around the ion, forming a potential energy barrier that electrons can then tunnel through. There are multiple approaches that enable the description of the ionized electron, but they are generally limited or models derived by analogy, and the status of the trajectories is unclear.

\vspace{\abstskip}
\noindent
This thesis analyses this trajectory language in the context of the Analytical R-Matrix theory of photoionization, deriving a trajectory model from the fundamentals, and showing that this requires both the time and the position of the trajectory to be complex. I analyse this complex component of the position and I show that it requires careful handling: of the potentials where it appears, and of the paths in the complex plane that the trajectory is taken through.

\vspace{\abstskip}
\noindent
In this connection, I show that the Coulomb potential of the ion induces branch cuts in the complex time plane that the integration path needs to avoid, and I show how to navigate these branch cuts. I then use this formalism to uncover a kinematic mechanism for the recently discovered (Near-)Zero Energy Structures of above-threshold ionization.

\vspace{\abstskip}
\noindent
In addition, I analyse the generation of high-order harmonics of the driving laser that are emitted when the photoelectron recollides with the ion, using a pair of counter-rotating circularly polarized pulses to drive the emission, both in the context of the conservation of spin angular momentum and as a probe of the long-wavelength breakdown of the dipole approximation.











\chapter*{Resumen}

Esta tesis explora la dinámica de la ionización de electrones inducida por un campo externo fuerte y de longitud de onda larga, y hace una descripción, basada en trayectorias semiclásicas, de las interacciones que ocurren entre el electrón y el ion una vez que el electrón aparece en el continuo.

\vspace{\abstskip}
\noindent
En general, la fotoionización puede describirse en términos de trayectorias semiclásicas en el régimen de tunelaje óptico: cuando un átomo interactúa con un campo fuerte y de frecuencia baja, se genera un potencial lineal que oscila con el campo, lo cual una barrera de energía potencial bajo la cual pueden salir el electrón. Una vez fuera del átomo, existen múltiples métodos para describir al electrón con base en las trayectorias clásicas del mismo sistema, pero en general dichos métodos son modelos que funcionan con base en analogías y sin sustento en la ecuación de Schrödinger que subyace al sistema, y la naturaleza de dichas trayectorias no queda enteramente clara.

\vspace{\abstskip}
\noindent
En esta tesis, analizo este tipo de modelos mediante el método analítico de la matriz R (analytical R-matrix theory), que permite la derivación de un modelo basado en trayectorias a partir de la ecuación de Schrödinger del sistema. Dentro de este formalismo, en general, la trayectoria del electrón tiene una componente imaginaria tanto de la posición como del tiempo. Un análisis detallado de esta componente imaginaria, y del comportamiento del potencial atómico cuando se incluye la misma, demuestra que se requiere cuidado con su uso para evitar divergencias y cortes rama en el potencial. 

\vspace{\abstskip}
\noindent
En particular, el potencial coulombiano del núcleo deja cortes rama en el plano complejo del tiempo, en el cual se desenvuelve la curva de integración que representa a la trayectoria. Este contorno de integración debe evitar dichos cortes rama, lo cual restringe el tipo de contornos que pueden usarse; para ello, presento un algoritmo que permite navegar los cortes de forma automática. De este algoritmo surge naturalmente una descripción cinética de estructuras a muy baja energía (Near-Zero Energy Structures), que fueron observadas recientemente en experimentos de ionización con campos fuertes con longitudes de onda en el infrarrojo.

\vspace{\abstskip}
\noindent
Adicionalmente, en esta tesis exploro la generación de armónicos de alto orden, que se emiten cuando el fotoelectrón regresa al ion y emite su energía cinética al recombinar con el hueco electrónico que dejó al salir del átomo. Más en específico, examino la generación de armónicos bajo dos campos circulares contra-rotantes, como un medio para entender la conservación de momento angular de spin durante el proceso, y para examinar el comportamiento en el límite de longitud de onda larga, donde se observa un fallo de la aproximación~dipolar.


































